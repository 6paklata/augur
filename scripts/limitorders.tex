\documentclass[floatfix,reprint,nofootinbib,amsmath,amssymb,epsfig,pre,floats,letterpaper,groupedaffiliation]{revtex4}

\usepackage{amsmath}
\usepackage{amssymb}
\usepackage{graphicx}
\usepackage{epstopdf}
\usepackage{dcolumn}
\usepackage{bm}
\usepackage{listings}
\usepackage{xcolor}
\usepackage{inconsolata}

\colorlet{punct}{red!60!black}
\definecolor{background}{HTML}{EEEEEE}
\definecolor{delim}{RGB}{20,105,176}
\colorlet{numb}{magenta!60!black}

\lstdefinelanguage{generic}{
    basicstyle=\fontsize{8}{10}\ttfamily,
    showstringspaces=false,
    breaklines=true,
    literate=
     *{0}{{{\color{numb}0}}}{1}
      {1}{{{\color{numb}1}}}{1}
      {2}{{{\color{numb}2}}}{1}
      {3}{{{\color{numb}3}}}{1}
      {4}{{{\color{numb}4}}}{1}
      {5}{{{\color{numb}5}}}{1}
      {6}{{{\color{numb}6}}}{1}
      {7}{{{\color{numb}7}}}{1}
      {8}{{{\color{numb}8}}}{1}
      {9}{{{\color{numb}9}}}{1}
      {:}{{{\color{punct}{:}}}}{1}
      {,}{{{\color{punct}{,}}}}{1}
      {\{}{{{\color{delim}{\{}}}}{1}
      {\}}{{{\color{delim}{\}}}}}{1}
      {[}{{{\color{delim}{[}}}}{1}
      {]}{{{\color{delim}{]}}}}{1},
}

\newcommand{\beq}{\begin{equation}}
\newcommand{\eeq}{\end{equation}}
\newcommand{\e}{\mathrm{e}}
\newcommand{\la}{\langle}
\newcommand{\ra}{\rangle}

\begin{document}

\title{Off-chain Limit Orders for the LMSR/LS-LMSR}

\author{Jack Peterson\\\emph{www.augur.net}}

\maketitle

The user wants to make a buy limit order on outcome $i$.  The user enters:
\begin{itemize}
\item Limit price $p(q_i)$.
\item Total number of shares to buy, $N$.
\item Price cap for the stop order, $\xi$.  This is the maximum price the user is willing to accept for this order.
\end{itemize}
What is the maximum number of shares ($n$) the user can buy such that the price $p(q_i+n)$ does not rise above the price cap ($\xi$)?
\beq\label{eq:solve_for_n}
p\left(q_i+n\right)=\xi
\eeq
To solve Eq.~\ref{eq:solve_for_n} for $n$, the price function $p$ must be specified.  First, use the LMSR's simple price function:
\beq\label{eq:price_LMSR}
p\left(q_i\right)=\frac{\e^{\beta q_i}}{\sum_{j} \e^{\beta q_j}},
\eeq
making the substitution $\beta\equiv b^{-1}$ for readability.\footnote{$p(q_i)$ and $b(q_i)$ are written as univariate functions because only $q_i$ is varied here.}  Plug Eq.~\ref{eq:price_LMSR} into Eq.~\ref{eq:solve_for_n} and rearrange to solve for $n$:
\beq\label{eq:solve_for_n_LMSR}
n = -q_i + \frac{1}{\beta} \log\left(\frac{\xi}{1-\xi} \sum_{j\ne i} \e^{\beta q_j}\right).
\eeq
If $n \ge N$, this is just a stop order: it converts to a market order, is completely filled by the automated market maker, and nothing further happens.  If $n < N$, a market order for $n$ shares is submitted and filled by the market maker (bringing the price to $\xi$).  This leaves $N-n$ total shares in the user's limit order.  The order remains open until the market maker's price again drops to the limit price $p(q_i)$.

The LS-LMSR's price function can also be used, although since it is more complicated there is not a closed-form expression for $n$ like Eq.~\ref{eq:solve_for_n_LMSR}.  The LS-LMSR's price function is
\beq\label{eq:price_LSLMSR}
p\left(q_i\right) = \alpha \log\left(\sum_j \e^{q_j/b(q_i)}\right) + \frac{\displaystyle \e^{q_i/b(q_i)} \sum_j q_j - \sum_j q_j \e^{q_j/b(q_i)}}{\displaystyle\sum_j q_j \sum_j \e^{q_j/b(q_i)}},
\eeq
where $b\left(q_i\right) \equiv \alpha \sum_j q_j$.  Plug Eq.~\ref{eq:price_LSLMSR} into Eq.~\ref{eq:solve_for_n}:
\beq
b\left(q_i + n\right) = b\left(q_i\right) + n
\eeq
\beq\label{eq:solve_for_n_LSLMSR}
\alpha \log\left(\e^{\frac{q_i + n}{b(q_i)+n}} + \sum_{j\ne i} \e^{\frac{q_j}{b(q_i)+n}}\right) +
\frac{
     \displaystyle \e^{\frac{q_i+n}{b(q_i)+n}} \sum_{j\ne i} q_j - \sum_{j\ne i} q_j \e^{\frac{q_j}{b(q_i)+n}}
	}{
	\displaystyle \left(n+\sum_j q_j\right)
	\left(\e^{\frac{q_i+n}{b(q_i)+n}} + \sum_{j\ne i} \e^{\frac{q_j}{b(q_i)+n}}\right)
	} = \xi
\eeq
Eq.~\ref{eq:solve_for_n_LSLMSR} is then numerically solved for $n$.

\subsection*{Example implementation (Matlab)}
\begin{lstlisting}[language=generic]
% price cap
xi = 0.3;

% LMSR
beta = 1;
q = 10*ones(1,5);
i = 1;
qj = [q(1:i-1) q(i+1:end)];
n = -q(i) + log(xi*sum(exp(beta*qj))/(1 - xi)) / beta
q(i) = q(i) + n;
p_lmsr = exp(beta*q) / sum(exp(beta*q))

% LS-LMSR
clear n
a = 0.0079;
q = 10*ones(1,5);
i = 1;
qj = [q(1:i-1) q(i+1:end)];
F = @(n) a*log(exp((q(i) + n)/a/(n + sum(q))) + sum(exp(qj/a/(n + sum(q))))) + ...
    (exp((q(i) + n)/a/(n + sum(q)))*sum(qj) - sum(qj.*exp(qj/a/(n + sum(q))))) / ...
    ((n + sum(q))*(exp((q(i) + n)/a/(n + sum(q))) + sum(exp(qj/a/(n + sum(q)))))) - xi;
n0 = fsolve(F, 0.05)
q(i) = q(i) + n0;
b = a*sum(q);
p_lslmsr = a*log(sum(exp(q/b))) + ...
    (exp(q/b)*sum(q) - sum(q.*exp(q/b))) / sum(q) / sum(exp(q/b))
\end{lstlisting}

\newpage

\subsection*{Example implementation (JavaScript)}
\begin{lstlisting}[language=generic]
#!/usr/bin/env node

var Decimal = require("decimal.js");
var fzero = require("fzero");

// MSR parameters
var q = [new Decimal(10),      // outcome 1 shares
         new Decimal(10),      // outcome 2 shares
         new Decimal(10),      // outcome 3 shares
         new Decimal(10),      // outcome 4 shares
         new Decimal(10)];     // outcome 5 shares
var i = 1;                     // outcome to trade
var a = new Decimal("0.0079"); // LS-LMSR alpha
var xi = new Decimal("0.3");   // price cap

// LS-LMSR objective function (Eq. 6)
function f(n) {
    n = new Decimal(n);
    var numOutcomes = q.length;
    var qj = new Array(numOutcomes);
    var sum_q = new Decimal(0);
    for (var j = 0; j < numOutcomes; ++j) {
        qj[j] = q[j];
        sum_q = sum_q.plus(q[j]);
    }
    qj.splice(i, 1);
    var q_plus_n = n.plus(sum_q);
    var b = a.times(q_plus_n);
    var exp_qi = q[i].plus(n).dividedBy(b).exp();
    var exp_qj = new Array(numOutcomes);
    var sum_qj = new Decimal(0);
    var sum_exp_qj = new Decimal(0);
    var sum_qj_x_expqj = new Decimal(0);
    for (j = 0; j < numOutcomes - 1; ++j) {
        sum_qj = sum_qj.plus(qj[j]);
        exp_qj[j] = qj[j].dividedBy(b).exp();
        sum_exp_qj = sum_exp_qj.plus(exp_qj[j]);
        sum_qj_x_expqj = sum_qj_x_expqj.plus(q[j].times(exp_qj[j]));
    }
    return a.times(q[i].plus(n).dividedBy(b).exp().plus(sum_exp_qj).ln()).plus(
        exp_qi.times(sum_qj).minus(sum_qj_x_expqj).dividedBy(
            q_plus_n.times(exp_qi.plus(sum_exp_qj))
        ).minus(xi)
    );
}

console.log(fzero(f, [0.001, 10]).solution);
\end{lstlisting}

\end{document}
